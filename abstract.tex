\begin{abstract}

Alzheimer's disease (AD), is a chronic neurodegenerative disease that usually starts slowly and gets worse over time. It is the cause of 60\% to 70\% of cases of dementia. There is growing interest in identifying brain image biomarkers that help evaluate AD risk pre-symptomatically. High-dimensional non-linear pattern classification methods have been applied to structural magnetic resonance images (MRI's) and used to discriminate between clinical groups in Alzheimer’s progression. 
Using Fluorodeoxyglucose (FDG) positron emission tomography (PET) as the preferred imaging modality, this thesis develops two independent machine learning based patch analysis methods and uses them to perform six binary classification experiments across different (AD) diagnostic categories. Specifically, features were extracted and learned using dimensionality reduction and dictionary learning \& sparse coding by taking overlapping patches in and around the cerebral cortex and using them as features. Using AdaBoost as the preferred choice of classifier both methods try to utilize \FDGPET~as a biological marker in the early diagnosis of \Alzheimers. Additional we investigate the involvement of rich demographic features (\apoe{3}, \apoe{4} and Functional Activities Questionnaires (FAQ)) in classification. The experimental results on Alzheimer's Disease Neuroimaging initiative (ADNI) dataset demonstrate the effectiveness of both the proposed systems. The use of \FDGPET~may offer a new sensitive biomarker and enrich the brain imaging analysis toolset for studying the diagnosis and prognosis of AD.
 
{\bf Keywords:} Alzheimer’s disease, dimensionality reduction, dictionary learning and sparse coding, Patch Analysis based
Sparse-coding System.


\end{abstract}
