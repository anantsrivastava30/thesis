\chapter{Discussion and Limitations}
Both the frameworks are are patch based differing in the extraction stage. The use of stochastic coordinate coding has been shown to work great with 3D surface data like MRI images ~\citep{zhang2016applying}. to model this system in voxel based 3d scans it also required that we consider 3d image patches. \citep*{zhang2016applying} and \citep{lin2014stochastic} explain in there work the computation overhead of SCC. Considering our $ 10\times10\times10 $ patches with a $ 332 $ sample size and $ 1000 $ number of features and our effective matrix size is $ 332000 \times 1000 $. This matrix has 320 million entries and conducting six experiments could be extremely challenging. We decided to pool the $ 10\times10\times10 $ matrix by a $ 2\times2\times2 $ window and down sample our patch features to $125$. This might cause unexpected variations in results. 

Our work purely concentrates on the usage of AdaBoost for classification. Our aim was to choose Adaboost as a classifier and find the best configuration to efficiently perform classification. FDG-PET image is 3 dimensional data, conforming to a voxel-wise structure of the brain. In the past SVM has been the choice of classifier for classifying clinical groups in (AD) progression so has been classifiers like incomplete random forests, \citep{lu2017early}. We choose Ada boost as it has been used before in our lab~\citep{zhang2016hyperbolic}, tweaking some other classifier may have given better performance.  

This project includes experimentation purely on the baseline (first visit) data from ADNI2. Integration of data collected from later visits in time can further help analyse the dataset. Classification with time, (i.e. if a subject is diagnosed as CU (healthy), and in a later visit diagnosed with MCI(Early or Late)) successful prediction on converters and non-converters with neural networks is a technique yet to be explored.

Since our optimal configurations were achieved by a brute force method, there is a possibility of our results improving further. Further experimentation with neural network configurations may help achieve even better classification accuracies.