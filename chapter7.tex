\chapter{Conclusion}
In this thesis, we present two frameworks that combines three dimensional voxel statistics with machine learning and  dictionary learning \& sparse coding to deal with high dimensional features before classification. We applied an AdaBoost to classify different AD stages. Our comprehensive experiments showed the effectiveness of patch based methods in three dimensional Positron Emission Tomography (PET). We obtained $ ~96 \% $ classification with voxel + demographic statistics.
Both frameworks perform well with experiments with high group separation. With experiments having low group separation both the algorithms struggled to some extent. This indicates that there is significant metabolic change in AD vs. CU but in other stages the metabolic change may be unpredictable to some extent. 

We hope our work sheds light on the utilization of PET images as biomarker information in classifying Non-AD classes with a greater accuracy. It also invokes the use of varied multiple features in the diagnosis of Alzheimer's via some clinical group classification. In the future, we plan to apply our systems to other cortical and sub-cortical regions in the brain, more specifically to design a better ROI based feature selection method so as to identify regions in the cortex and sub-cortex responsible for cognitive decline.